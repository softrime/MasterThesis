% !TEX root = ../thesis.tex

%TC:ignore

\begin{acknowledgements}
  在接近三年的研究生生活即将结束之际,在此我首先要感谢我的导师俞凯教授。感谢俞凯老师
  接受我作为SpeechLab的一员,带我走进智能语音的世界。研究生三年的时间,我从一个基础较差
  的计算机毕业生,到对一个领域有了自己的经验和理解,这其中的成长和俞凯老师平日里对我们的
  教导和树立的榜样密切相关。知识的增加和技能的进步是我们走向社会的一把钥匙,但更为
  重要的则是在实验室期间养成的做人,做事和做学问的严谨态度,它们将伴随我们的一生并在时间的
  大河中形成一阵阵或大或小的波浪不断前行,这是俞老师带给我的最宝贵的财富。也要感谢吴老师和茅老师
  对我在科研和生活上的帮助和鼓励,带给我不光学问上的增加,还有情商的成长。

  同样感谢SpeechLab的各位师兄师姐师弟师妹,实验室良好的科研和交流环境让我在专注于自己的方向时
  也能不断学习到其他方向的知识和最新成果。感谢合成组的陈博师兄和已经毕业的赖家豪师兄,在我入门
  语音合成和语音转换时提供了巨大而又无私的帮助,尤其是陈博师兄与我每次讨论问题时都会认真且不计麻烦
  的询问细节和提出建议,这让我十分感动。感谢实验室和思必驰公司提供的超算,为我提供充足的计算资源。
  感谢实验室与我同届的郭嘉琪,李豪,赵子健,陈烨斐,黄明坤,陈宽,陈志在研究生期间带给我的快乐回忆,
  人生中可能是最后的学生阶段与你们一起度过甚是有幸。

  最后感谢我的所有家人,朋友以及同学们。是你们在我过去的生活中出现的一点一滴和无数不经意的眼神和话语,
  成为了我脑海中所有的记忆,塑造了现在的我,影响着我的每一次决定。感谢我的女朋友范锐,在我研究生期间
  陷入低谷的时候对我无限的鼓励和陪伴,让我不断鼓起勇气坚持下去。感谢我的父母,你们对我所有选择的尊重和
  支持是我每一次不顾后果地突破自我的信心,也是我努力的动力。



\end{acknowledgements}

%TC:endignore
