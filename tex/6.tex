\chapter{全文总结}
非平行语音转换是语音转换中非常重要的一个方向。本文工作主要基于CycleGAN语音转换方法,
围绕不同的非平行语音转换任务展开。

首先对于标准的非平行语音转换任务,
本文提出半优化循环一致性生成对抗网络(Semi-optimized CycleGAN)和基频辅助特征的高音质非平行语音转换系统。
该系统使用梅尔频谱作为声学特征,WaveNet模型作为声码器。经过分析,本文对标准CycleGAN模型的
训练过程进行改进,去除了其中影响生成器训练的更新过程,使得生成器在每个周期中只有其中一个被优化;
同时针对转换中音调错误的问题,提出使用基频辅助特征来提升生成器对基频的隐式表达能力。实验表明,
半优化CycleGAN不但可以显著的减小模型训练过程中在测试集上的梅尔频谱距离,也可以减小转换基频与
目标基频之间的误差。在自然度和相似度的主观实验中,半优化CycleGAN和基频辅助特征都对转换语音有着较好的提升。
尤其在男转男的说话人上,发音错误问题较为严重,基频辅助特征使其MOS分数有了明显的提升。实验也比较了
不同大小训练数据量对模型性能的影响,发现在训练集在最少为500句话的时候都可以得到较高的音质。

对于难度更高的跨语种语音转换任务,本文提出了基于说话人特征的语音转换模型。首先用说话人识别的
语料训练一个说话人识别模型,然后将说话人识别模型的最后一层去掉得到说话人特征提取器。将该提取器
与CycleGAN模型进行结合,其作为判别器的前置网络,从输入的声学特征中提取说话人信息,来减少
语义信息对判别器的影响,从而提升转换效果。实验表明,对于语音转换的较少说话人而言,100个说话人
的训练集得到的说话人特征提取器也可以对语音转换说话人进行较好的分离。同时,说话人特征的引入不但可以
减小转换音频说话人特征与真实音频说话人特征之间的余弦距离,还可以使得模型训练中重构损失更稳定。
在主观测试中,引入说话人特征的方法在自然度和相似度上都要好于没有引入说话人特征的标准CycleGAN模型,
另一方面,使用语种无关的对抗训练方法而得到的语种无关的说话人提取器,可以略微提升转换语音的自然度和相似度。
但是,当前方法与真实语音在自然度和相似度上都还有不小的差距,一方面是由于生成语音的噪声较大,尽管语义相比
基线模型有所提升,但依旧不够清晰,转换模型仍有很大的提升空间;另一方面,声码器也是制约音质的一个因素,
将目前使用的跨语种语音转换方法应用到神经网络声码器中也是未来的重点研究方向。